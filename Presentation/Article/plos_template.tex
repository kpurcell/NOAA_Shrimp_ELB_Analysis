% Template for PLoS
% Version 1.0 January 2009
%
% To compile to pdf, run:
% latex plos.template
% bibtex plos.template
% latex plos.template
% latex plos.template
% dvipdf plos.template

\documentclass[10pt]{article}

% amsmath package, useful for mathematical formulas
\usepackage{amsmath}
% amssymb package, useful for mathematical symbols
\usepackage{amssymb}

% graphicx package, useful for including eps and pdf graphics
% include graphics with the command \includegraphics
\usepackage{graphicx}

% cite package, to clean up citations in the main text. Do not remove.
\usepackage{cite}

\usepackage{color} 

% Use doublespacing - comment out for single spacing
\usepackage{setspace} 
\doublespacing


% Text layout
\topmargin 0.0cm
\oddsidemargin 0.5cm
\evensidemargin 0.5cm
\textwidth 16cm 
\textheight 21cm

% Bold the 'Figure #' in the caption and separate it with a period
% Captions will be left justified
\usepackage[labelfont=bf,labelsep=period,justification=raggedright]{caption}


% Use the PLoS provided bibtex style
\bibliographystyle{plos2009}

% Remove brackets from numbering in List of References
\makeatletter
\renewcommand{\@biblabel}[1]{\quad#1.}
\makeatother


% Leave date blank
\date{}

\pagestyle{myheadings}
%% ** EDIT HERE **


%% ** EDIT HERE **
%% PLEASE INCLUDE ALL MACROS BELOW

%% END MACROS SECTION

\begin{document}

% Title must be 150 characters or less
\begin{flushleft}
{\Large
\textbf{Hypoxia effects on the spatial dynamics of a coastal fishery}
}
\vspace{10mm}
% Insert Author names, affiliations and corresponding author email.
\\
K.M. Purcell$^{1,\ast}$, 
J.K. Craig$^{1}$, 
J.M. Nance$^{2}$,
M.D. Smith$^{3}$
\vspace{5mm}
\\
\bf{1} Beaufort Laboratory, National Marine Fisheries Service, Beaufort, NC, USA
\\
\bf{2} Galveston Laboratory, National Marine Fisheries Service, Galveston, TX, USA
\\
\bf{3} Nicholas School of the Environment, Duke University, Durham, NC, USA
\\
$\ast$ E-mail: Corresponding kevin@kevin-purcell.com
\end{flushleft}

% Please keep the abstract between 250 and 300 words
\section*{Abstract}

% Please keep the Author Summary between 150 and 200 words
% Use first person. PLoS ONE authors please skip this step. 
% Author Summary not valid for PLoS ONE submissions.   
%\section*{Author Summary}

\section*{Introduction}
Marine and coastal ecosystems are being increasingly affected by patterns of environmental hypoxia which have been documented in over 400 marine systems affecting greater than 240,000 $km{}^{2}$ \cite{diaz_spreading_2008}.  Environmental hypoxia, a condition of \textless 2 $ mg{} L^{-1} $ dissolved oxygen (DO), is generally a product of excess consumption of oxygen through respiration and chemical processes relative to the rate of production.  Hypoxia, especially in shallow water coastal systems, is more common in areas with high nutrient inputs, longer residence times and physical stratification, characteristics common to coastal and estuarine systems \cite{rabalais_global_2009}.  Currently, the northern Gulf of Mexico (Gulf) experiences one of the largest seasonal hypoxia events in the western hemisphere \cite{rabalais_beyond_2002}.  The areal extent of Gulf hypoxia coverage has been estimated annually since 1985 and has shown significant variation exceeding 20,000 $ km^2 $ in recent years (Rabalais ?, Turner et al. 2008 via aggregations: Figure 1).  Hindcasting models and paleo-indicator studies show indications of low oxygen stress within the Gulf beginning in the early 1900s becoming increasingly severe during the 1960s and 1970s \cite{justic_forecasting_2007, rabalais_sediments_2007, greene_multiple_2009}.  The causal mechanisms for natural and anthropogenically mediated hypoxia have been extensively discussed in the literature \cite{bianchi_science_2010, rabalais_dynamics_2010, zhang_natural_2010} with a clear consensus that Gulf hypoxia is driven by increased eutrophication subsequent to the Mississippi-Atchafalaya River systems which is directly adjacent to the Louisiana coastal shelf.  The increased nutrient loading to the coastal shelf and the vertical stratification that result from seasonal river flow creates ideal conditions for the annual development of bottom water hypoxia \cite{rabalais_beyond_2002, bianchi_science_2010}.  The hypoxic zone (region with \textless 2.0 $ mg{} L^{-1} $ DO) is highly variable spatially, depending on prevailing weather patterns and coastal currents (XXX).

The Gulf shrimp fishery is an open access, otter trawl fishery occurring throughout the coastal shelf of the Gulf of Mexico Exclusive Economic Zone (EEZ), from southern Florida to the Texas-Mexico border.  The majority of fishing effort occurs in the northwestern Gulf and predominantly targets three Penaeid shrimp species: the Brown \textit{Farfantepenaeus aztecus} and to a lesser extent the White \textit{Litopenaeus setiferus} and Pink \textit{Farfantepenaeus duorarum} shrimp.  The fishery began in 1917 with an inshore artisanal fleet focusing on a single species \textit{L. setiferus} and has grown to a large commercialized fleet in excess of 20,000 vessels (Condrey and Fuller 2005, Diamond 2004).  Historically, Gulf shrimp landings provided roughly 73\% percent of the total shrimp landings in the US (Iverson and Martin 2008) and the fishery was valued at \textless \$500M USD annually, making it one of the most profitable fisheries in the US \cite{nance_effort_1993, diamond_bycatch_2004}. 

The region of the northwestern Gulf associated with the Mississippi-Atchafaylaya river plum has been referred to as the "Fertile Fishery Crescent" (Gunter 1963 via NGOMEX) and has supported some of the most productive fishing grounds within U.S. coastal waters (Chesney et al 2000).  The spatial overlap between this fertile fishing ground and the spatial extent of seasonal hypoxia has led to a number of studies focused on the interaction between nektonic populations and coastal hypoxia (Caddy 1993, Breitburg et al. 2009, Rose et al. 2009, Essington and Paulsen 2010).  More specifically, Craig et al. (2005) examined the spatial distribution of F. aztecus in the northwestern Gulf of Mexico and reported that wide spread hypoxia (\textgreater 20,000 $ km^{2}$) results in a $\sim$25\% reduction in shrimp habitat coupled with shifts in spatial distribution both inshore and offshore of hypoxic regions.  Craig and Crowder (2005) reported that both \textit{F. aztecus} and Atlantic Croaker \textit{Micropogonias undulates} show clear patterns of avoidance in very low oxygenated waters and increased abundance at moderate (1.6-3.7 $ mg{} L^{-1} $) levels of dissolved oxygen, indicating possible associations with hypoxia edge habitats.  More recently, Craig \cite{craig_aggregation_2012} examined two 4000-5000 $ km^{2} $ regions of the Louisiana coastal shelf, and found that for \textit{F. aztecus} and a number of finfish species dissolved oxygen avoidance thresholds were on average 1-3 $ mg{} L^{-1} $ and that all species showed strong aggregations within 1-3 $ km $ of that edge of hypoxia.  

While increased patterns of spatial aggregation could serve as a spatially mediated response to hypoxia stress it could also facilitate the possibility of increased fishery bycatch or patterns of hyperstability within the fishery that could compromise management and threaten the long-term resilience of populations.  Fishery bycatch, especially within the trawl fisheries, has been identified as a principle threat to global commercial fish stocks (Davies et al. 2009, Kumar and Deepthi 2006).  Since the 1980s, Gulf shrimp trawl fishery bycatch has been identified as a key management issues and has undergone considerable scrutiny (Crowder and Murawski 1998).  The issue of fishery bycatch within the Gulf shrimp fishery was of significant threat to justify an amendment to the Magnuson Fishery Conservation and Management Act that requires research focused on issues of bycatch and lead to implementation of an observer program within the fishery \cite{diamond_bycatch_2004}.  Gallaway \cite{gallaway_evaluation_2003} indicated that issues of high bycatch rates within the shrimp fishery dictate a richer understanding of the temporal and spatial trends in fishing effort and how those trends are affected by alterations in the abiotic environment.  

In traditional fishery stock assessment models it is assumed that CPUE is proportional to resource abundance or population size (Quinn and Deriso 1999, Erisman et al. 2011) and this relationship is essential to sound management decisions.  Hyperstability is a disproportional condition in which catch per unit effort (CPUE) displays a non-linear relationship with resource abundance (Harley et al 2001).  This condition can facilitate overestimates of resource biomass and underestimates of fishery mortality, which can compromise the management of fishery stocks (Crecco and Overholtz 1990, Erisman et al. 2011, MORE CITATIONS).  The spatial and temporal allocation of fishing effort dictates the relationship between CPUE and resource abundance and therefor any changes in the spatial distribution of effort could create patterns of hyperstability that could obfuscate the effects of hypoxia stress on coastal fishery populations.  

In this study we examine the relationship between the spatial allocation of fishing effort within the northwestern Gulf of Mexico shrimp fishery and a number of important environmental and economic variables affecting the spatial region of the fishery.  We evaluate the hypothesis that hypoxia induced shifts in the spatial distribution of coastal nekton could drive the changes in the spatial allocation of fishing effort within the Gulf shrimp fishery.  Further, we discuss the implications of changes in the allocation of fishery effort with respect to the future management the Gulf shrimp fishery and the ecology of the northwestern Gulf region.  

% Results and Discussion can be combined.
\section*{Results}
GAM regression models were developed to examine the role of environmental heterogeneity on fishing effort for both the Louisiana and Texas coastal shelves employing both non-spatial and spatial model formulations.  Comparison of model formulations showed similar significance patterns for all parametric and 1 dimensional smooth terms for both model formulations.  Spatial model formulations were marginally superior to non-spatial formulations for both regions in terms of GCV score (Table 1).  Evaluation of model residuals showed no significant departures from the underlying assumptions of normality and variance homogeneity.

The non-spatial regression models for fishing effort on the Louisiana and Texas shelves had significant effects for all parametric model parameters with the exception of a single year (2005) in the Louisiana model (Table 1).  Dissolved oxygen had a marginal positive effect on the quantity of fishing effort on the Louisiana shelf at values \textless 2 $mg{}L^{-1}$ and displayed a negative effect in waters \textgreater  5 $mg{} L^{-1}$ (Figure 1A).  On the Texas shelf dissolved oxygen had a largely negative effect on effort at values \textless  2.5 $mg{} L^{-1}$ and \textgreater  5.5 $mg{}L^{-1}$ (Figure 1B).  Both regions displayed positive effects of dissolved oxygen on fishing effort between 2.5 - 4.5 $mg{}L^{-1}$ (Figure 1A-B).  In terms of depth, fishing effort in Louisiana showed a positive effect in waters shallower than 18 $m$ and negative effects in waters \textgreater 20 $m$ (Figure 1C) while Texas waters displayed strongly negative effects in waters \textless 20 $m$ and only limited positive effects in deeper waters (Figure 1D).  Market price for shrimp ($pPND$) showed limited effect overall on fishing effort in Louisiana (Figure 1E), while the Texas model showed some negative effects at values between 2-3 USD and positive effects between 3-4 $USD$ (Figure 1F).  Julian day was included in our model to reflect some of the interannual differences in the shrimp fishery between our two regions.  The effects of Julian day on fishing effort in Louisiana waters were positive before early July (\textless day 184), later in the year effects were consistently negative.  By comparison the effect of Julian day on the Texas shelf was strongly negative earlier than mid July ($\sim$day 196) with a positive effect peak in late July ($\sim$day 205) and negative effects through August.  
 
The utility of the geospatial regression models is their ability to explore the effects of space, and spatially variable environments, on a response variable, in this case the distribution of shrimping effort.  The effect of spatial location on the Louisiana shelf, based on average conditions for all other covariates, showed the highest fishing effort east of the river delta and in offshore waters between Atchafalaya Bay and the Louisiana bight (Figure 2A).  By contrast, spatially explicit fishing effort predictions followed a uniform structure strongly correlated with depth in Texas coastal waters.  Higher predicted shrimping effort was predicted inshore with effort declining with increasing depth (Figure2B).  The variable-coefficient terms included in our spatial model formulation allow us to examine spatial variability in the effects of dissolved oxygen concentration across both the Louisiana and Texas coastal shelves.  We identified significant positive and negative correlations with dissolved oxygen in spatial model formulations for both regions (Figure 2).  The Louisiana shelf model identified regions of significant postive correlation between DO and fishing effort in waters adjacent to Terrebonne Bay, inshore and west of Atchafaylaya Bay and in the southwesterly most region of the Louisiana fishing grounds (Figure 2A). Significant negative correlations between DO and total fishing effort were identified around the Mississippi River delta, the Louisiana Bight, and throughout much of the mid-shore and offshore waters from the eastern side of Atchafaylaya Bay to the western boarder of Louisiana waters (Figure 2A).  The spatial model for fishing effort on the Texas shelf, a model developed because of the regions relatively limited annual hypoxia showed a much more uniform distribution with negative correlations occuring in inshore waters  north of Galveston Bay to the mouth of Matagorda Bay (Figure 2B).  Positive correlations of DO and shirmping effort were identified in offshore waters for much of the Texas coast and covering the entire shelf in waters off Corpus Christi (Figure 2B).  A small region of negative correlations was observed in waters south of Corpus Christi extending to the southern most boundry of the US fishery (Figure 2B).  
 
To further elucidate the effect of annual hypoxia on the spatial allocation of fishery resources on the Louisiana and Texas coastal shelves, we conducted model simulations for both regions comparing high (22,300 $km^{2}$) and low (7100 $km^{2}$) aerial extents of hypoxia (Figure 3).  These values were chosen to contrast two ends of the hypoxia spectrum for years in our timeseries and are values upon which our models were trained.  Predicted surfaces for fishing effort allocation display dramatic spatial effects for the Louisiana shelf under high hypoxia conditions (Figure 4A) in which the highest predicted effort is east of the Mississippi River delta with regions of low effort predicted offshore of Terrebonne and Atchafalaya Bays.  By contrast, under low hypoxia conditions (Figure 4B) we observe the highest predicted effort west of the Mississippi River delta including waters adjacent to Louisiana bight and offshore of Terrebonne Bay.  Additionally, we observe more uniformity in the spatial distribution of shrimping effort across much of the western Louisiana shelf.  On the Texas coast, under high hypoxia conditions, we observe the highest effort inshore adjacent to Galveston and Matagorda Bays and in deeper waters off of Matagorda Bay (Figure 4C).  Interestingly, we also observed a region of very low predicted effort in Texas waters northeast of Galveston Bay (Figure 4C).  However, under low hypoxia condition we see little heterogeneity in the spatial distribution of predicted effort on the Texas shelf.  The highest predicted effort occurs in inshore waters adjacent to both Galveston and Corpus Christi Bays, with lower shrimping effort in deeper waters (Figure 4D).  
 
 
%These models were simulated based on fishing effort under different hypoxia conditions each representing different years 2008 (22.300 $ km^2 $) and 2009 (8000 $ km^2 $), however annual fishery effort was relatively similar in both years with 80,732 and 76,427 tow hours, respectively.  This similarity in total fishing effort between model simulations indicates that relative shifts in effort allocation may be informative.  For instance on the Louisiana shelf we see effort increases both east of the Mississippi River delta and offshore of the Atchafalaya Bay, these increases agree spatially, with predicted model effects of area of hypoxia on fishing effort (Figure 3B).  By contrast the allocation of fishing effort for the Texas shelf model displays little qualitative difference between these high and low hypoxia years, the primary difference consist of northward shifts in effort allocation primarily offshore of Galveston Bay (Figure 6 C, D), which also concurs with the model predictions (Figure 5B). 
 
%Models for average tow duration and tow density were conducted to indirectly assess tow quality and fishery behavior in the absence of corresponding landings information.  Both models showed significant effects for all  model parameters, with the exception fuel price for the average tow duration model and  certain years (enter years) for the tow density model in Texas waters (Table 3).  Model smooth relationships for average tow duration and tow density were qualitatively similar to total effort relationships for all smooth model parameters, and therefore we focus on a comparison of the predicted response surface for each model and their accompanying dissolved oxygen effects.  On the Louisiana coast our model predicts regions of high average tow duration around the mouth of the Mississippi River increasing as you move offshore along the eastern portion of the Louisiana shelf.  Additionally, a region of high average tow duration was predicted for inshore waters begining just west of Atchafalaya Bay for inshore waters of the western Louisiana shelf.  Dissolved oxygen was positively correlated with average tow duration predominatly in inshore waters  from the river delta to the far western boarder of Louisiana.  There was a large offshore region in which our model predicted low average tow durations and a negative relationship between dissolved oxygen and average tow duration (Figure 3A).  On the Texas shelf average tow durations were predicted to be highest adjacent to Galveston Bay with values decreasing as you move south and into deeper waters.  The model predicted positive correlations between dissolved oxygen and average tow duration offshore of Galveston Bay and in inshore waters adjacent to Corpus Christi Bay(Figure 3B).  Negative correlations were observed in inshore waters along much of the Texas coast and in the southern most reaches of the fishery along the Mexican boarder (Figure 3B).  For tow density we observed the highest predicted densities of tow east of the river delta with much of the Louisiana shelf showing only subtle variation (Figure 3C).  Positive correlations between dissolved oxygen and tow density were predicted around the river delta, in the Lousiana bight and in waters adjacent to Terrebonne Parish, LA.  Negative correlations were predicted adjacent to Terrebonne Bay just west of the Louisiana Bight, and o

%This languange was from the 1D smooth assessment of these models removed for the 2D smooth approach
Models for average tow duration and tow density were conducted to clarify fishery behavior in the absence of corresponding landings information.  Both models showed significant effects for all  model parameters, with the exception of fuel price for the average tow duration model and  certain years (2006-2008) for the tow density model in Texas waters (Table 3).  Model smooth relationships for average tow duration and tow density were qualitatively similar to total effort relationships for all smooth model parameters, except for dissolved oxygen.  The average duration of a tow on the Louisiana shelf was negatively associated with waters of \textless 2.0 $mg{} L^{-1}$ and \textgreater 5 $mg{} L^{-1}$ DO (Figure 4A).  On the Texas coast we saw negative effects of average tow duration in waters \textless $\sim$ 3 $mg{} L^{-1}$ and limited effects at higher levels of DO concentrations (Figure 4B).  In contrast to average tow duration, models for tow density showed positive effects for values of 2.5-3.5 $mg{} L^{-1}$ DO and negative effects for concentrations between 1-2 $mg{} L^{-1}$ and at values \textgreater 5 $mg{} L^{-1}$ DO (Figure 5A).  On the Texas shelf positive effects were observed for DO concerntrations between 2.5 and 4.5 $mg{} L^{-1}$, with negative effects on tow density between 6.5 and 7.5 $mg{} L^{-1}$ (Figure 5B).      


\section*{Discussion}
Our analysis of three fishing effort metrics, describing 6 years and $\sim 36,000$ fishery tow records, shows strong indications of a spatially mediated response by the Gulf shrimp fishery to hypoxia in the northwestern Gulf of Mexico.  We observed limited positive and negative effects of environmental and economic variables on the quantity of shrimping fishing effort in both regional models, however we did observe dramatic patterns in the spatial allocation of fishery resources when spatially explicit dissolved oxygen concentration was considered.  Previous studies have utilized these methods to examine distribution shifts in biological resources (Bacheler et al. 2010, Ciannelli et al. 2007, Bartolino et al. 2012), however, to the authors knowledge, this is the first application to the distribution of fishery resources with respect to environmental conditions.  

We observed several notable contrasts in the relationship of fishery effort to environmental and economic model parameters between the Louisiana and Texas models.  For instance depth displayed a strongly positive effect on fishing effort in the Louisiana model but displayed the opposite effect in Texas waters.  Additionally, the effect of Julian day in the early part of the year showed opposite effects between the two regions.  These differences in model effects reflect management differences between the two regional fisheries, specifically the early season closure of Texas waters \cite{klima_review_1982,klima_review_1984}.  More interestingly we observed similar positive effects between the two region models for moderate levels (2.5 - 4.5 $mg L^{-1]$)


The spatial response observed by the Gulf shrimp fishery, specifically an offshore shift along the western Louisiana shelf and an inshore and westward shift on the eastern shelf would seem to fit the results of previous studies focused on species distributional responses to hypoxia.  In Craig (2012) it was observed that the abundance of \textit{F. aztecus} on the Louisiana shelf was elevated at distances from 1-3 km from the edge of the hypoxic zone, additionally this study showed that positive catches were spatially distributed both inshore and offshore of the hypoxic zone.  Further, additional studies have documented clear patterns of low oxygen (< 3 mg l-1) avoidance behavior among populations of F. aztecus and other nektonic species (Craig et al. 2005, Craig and Crowder 2005).  While these studies temporally overlap our study by only a single year (2004) the apparent pattern of spatially mediated avoidance of low oxygen conditions does display a spatial similarity to the response of the fishery itself, and would seem to support our hypothesis that while coastal hypoxia does not seem to be negatively effecting landings (Breitberg 2009, Diaz and Solow 1999??, Chesney et al. 2001) there are clear patterns of spatial effects on the allocation of effort which could have both ecological and economics consequences.  

The quasi-experimental framework created in our study, through the model comparison between the Louisiana and Texas shelves, shows us several important findings.  First, we did see a negative effect of low dissolved oxygen on fishing effort in our analysis of the Texas shelf, this negative effect was observed in waters \textless 3 $mg L^{-1}$, and was not observed in the Louisiana model.  This finding was not surprising given that the low occurrence of hypoxia on the Texas shelf, recall that our data set had only a small number of hypoxic data points for the Texas region ($n$ = 35) in comparison to the Louisiana shelf ($n$ = 7085).  This difference in the magnitude of hypoxia would allow for easier avoidance of hypoxic conditions spatially, in Texas waters relative to Louisiana waters.  Additionally, we found that the market price for shrimp landings had positive effects on effort on the Texas shelf, specifically at values from 3-4 USD per pound.  This difference between the Louisiana and Texas fisheries would seem to fit a traditional fishery economics model, in which profit maximization of the firm, or the fisherman, dictates the behavior of the firm \cite{van_putten_theories_2012}.  Due to the Texas seasonal closure \cite{klima_review_1982,matlock_did_2010}, shrimp populations on the Texas shelves are typically of greater maturity and thereby greater size during the period of fishery operation (mid-June - mid-July), this coupled with differences in the size of fishery fleets between Texas ($n$ = 2988) and Louisiana ($n$= 9778)\cite{diamond_bycatch_2004} would favor a result in which more effort is expended on larger, and on average higher priced, shrimp landings relative to the Louisiana fishery.  Finally, the spatial shifts in fishing effort offshore and to the north and south of Galveston and Corpus Christi bays and a small positive region inshore adjacent to Matagorda Bay are qualitative distinct from those observed for the Louisiana shelf.  These findings, especially spatial shifts adjacent to Corpus Christi and Matagorda Bays were unexpected due to the variable-coefficient parameter being based on the aerial extent of hypoxia which typically does not extend south of Galveston Bay (Rabalais 2002).  While the magnitude and range of the positive effects along the Texas shelf are smaller than those on the Louisiana shelf it could indicate that the fishery responds to hypoxia in aggregate and not, as hypothesized, with regionally distinct behavior.  This interpretation is further supported by the fact that the spatial shifts observed do not seem to coincide with what is known about the incidence of hypoxia along the Texas coastal shelf (Osterman 2003).  Dimarco et al (2012) showed, based on 2007-2008 SEAMAP data, that the Mississippi-Atchafalaya river system is not the only causal mechanism for hypoxia on the Texas shelf, however the only river system that feeds directly to the coastal shelf and not through an embayment or an estuary is the Brazos River (Harper et al. 1991, Dimarco et al. 2012) which is located just south of Galveston Bay and relatively far from the positive effect shifts we observed adjacent to Matagorda or Corpus Christi Bays.

One criticism of our study design could be that without tow specific landings information, a reduction in fishing effort within or adjacent to low oxygen regions could be seen as ambiguous.  This ambiguity derives from the issue of hyperstability, a condition in which catch rates remain unchanged regardless of reductions in population abundance because the fishery is better able to target and harvest a given resource \cite{harley_is_2001,rose_hyperaggregation_1999}.  To address these issues we developed models based on two additional response variables: average tow duration and tow density to indirectly evaluate tow quality.  The underlying assumption for all of our statistical models is that the parties (fisherman) being examined are following a profit maximization scheme and their behaviors are dictated by an underlying imperative to maximize returns while minimizing effort \cite{branch_fleet_2006,van_putten_theories_2012}.  Given this assumption and the knowledge that shrimp fisherman possess knowledge about landings accumulation rates based on readings from a tensiometer, a device that measures tension on the trawl rigging, our finding for the Louisiana self that at dissolved oxygen values < 2.5 $mg L^{-1}$ average tow duration is decreasing could indicated reduced tow quality in hypoxic conditions.  We also found evidence of higher tow density within grid cells that have dissolved oxygen levels 2.5- 3.5 $mg L^{-1}$ along the Louisiana shelf, possibly indicating increased searching behavior within low oxygen habitat.  To a lesser degree similar patterns were observed in low oxygen conditions within the Texas fishery indicating that these fishery behaviors may indeed be a response to catch indicators and not solely a location driven response to previous knowledge concerning the annual region over which hypoxia occurs.

If the spatial dynamics of the shrimp fishery are changing in response to seasonal hypoxia, increased opportunities for new biological interaction with non-target species could be created.  Nance et al. (1998) quantified levels of bycatch biomass within the shrimp trawl fishery and found that on average 29 kg of bycatch, or roughly 1300 organisms per hour, were captured by Gulf shrimp trawl fisherman.  While management initiatives have been developed since this time, such as bycatch reduction devices (BRDs), their efficacy has had diminishing returns over time \cite{diamond_bycatch_2004,foster_status_2004}.  It would seem, given that previous studies have reported that shrimp and nektonic species aggregate within 1-3 km of the edge of hypoxic zones \cite{craig_aggregation_2012,craig_hypoxia-induced_2005}, that any attempt by the fishery to modify its spatial deployment in response to or in anticipation of hypoxia effects, will increase opportunities for bycatch and additional non-target species interactions to develop.

While our analysis does not permit a quantification of the spatial extent of the Gulf shrimp fishery, differences in the spatially explicit predicted fishing effort on both the Louisiana and Texas shelves does imply differences in the spatial patterns over which we predicted high levels of fishing effort.  A visual comparison of the predicted spatial effort landscapes during a high and low hypoxia year shows a clear reduction in regions of high fishing effort.  This result would seem to concur with previous studies that showed that in years with spatially extensive bottom water hypoxia (\textgreater 20,000 $km^{2}$) reductions in brown shrimp (\testit{F. aztecus}) habitat ($\sim$ 25\%) can be observed along the Louisiana shelf \cite{craig_spatial_2005}.  Without a quantifiable measure of the spatial extent of the fishery it is difficult to make conclusions concerning the economic ramifications of spatial shifts in fishery operations.  While we observe changes in the spatial regions at which the fishery operates in response to hypoxia the possibility that an overall reduction in the spatial extent of fishery operations could mitigate such economic consequences.  Therefore, while our findings are strictly suggestive they do seem to indicate that further examination of the spatial extent of fishery operations could be warranted, especially considering the economic and ecological ramifications such changes would have on both the fishery and the Gulf of Mexico ecosystem.

In summary, our analysis indicates that the spatial patterns of hypoxia on the coastal shelf of the northwestern Gulf of Mexico has a significant effect on the spatial dynamics of fishery resources.  In addition, we found that levels of dissolved oxygen have significant non-linear effects on patterns of fishery behavior (tow duration, and tow number) and these patterns suggest reduced catch and increased searching behavior within low oxygen or hypoxia edge habitat, based on a profit maximization model of fishery behavior.  Recently, the effect of seasonal hypoxia on commercial fisheries within the Gulf of Mexico has been pejoratively dismissed \cite{cowan2009ugly}, justified by the consistent increases observed in fishery landings (Chesney et al. 2000, Chesney et al 2000b, De Mutsert et al 2008) and a general lack of fishery losses in economic rents, or profits (NMFS document??).  Additionally, it is argued that the detrimental effects of nutrient inputs into hypoxic coastal systems are often mitigated or offset by the inherent flexibility of nektonic populations (Breitburg 2002).  However, the dynamics of fisheries is such that detrimental effects do not always manifest as direct loss in rents (Examples??).  Previous studies have shown that in nektonic species the impact of hypoxia is primarily observed not in population dynamics but rather in spatial shifts in abundance \cite{craig_aggregation_2012,craig_small_2013} other systems?, supporting the theory that vagile species will move to avoid stressful and deleterious environments if possible.  Our observed changes in the spatial dynamics of the Gulf shrimp fishery constitute substantive indirect effects of hypoxia and eutrophication.  Further, changes in the spatial dynamics of the shrimp fishery could have direct impacts on fishery rents through increased searching behaviors within low oxygen habitats and possibly through increased travel distances to harvest locations.  In addition, to these direct impacts there exists increased opportunities for indirect effects of hypoxia by way of increased bycatch rates due to spatial aggregation and hypoxia mediated dispersal patterns for other non-target nektonic species.  In conclusion, while we historically observed consistent patterns of fishery landings within coastal fisheries in the face of increased occurrence of coastal hypoxia, it is important to recognize that spatially mediated responses to hypoxia will have effects on the ecology and management of fishery resources especially in light of an increased focus on ecosystem based management practices.  

% You may title this section "Methods" or "Models". 
% "Models" is not a valid title for PLoS ONE authors. However, PLoS ONE
% authors may use "Analysis" 
\section*{Methods}
\subsection*{Shrimping Effort}
In 1999 an electronic log book (ELB) program was implemented in the northern Gulf of Mexico to measure the spatial and temporal distribution of shrimping effort in the Gulf shrimp otter trawl fishery \cite{gallaway_description_2003}.  A random sample of shrimp vessels were selected on a trimester basis (Jan-Apr, May-Aug, Sep-Dec) and instrumented with an electronic logbook (ELB) that recorded geographic position at 10 minute intervals and total time for individual shrimp tows.  Individual tows were identified based on changes in vessel speed which are typically 4-5 times slower (2-3 knots) when the shrimp nets are deployed than during other activities (i.e., steaming; 8-10 knots). Vessels with at least one landing in a given trimester of the previous year were used as the sampling universe, and vessels were drawn at random in proportion to their relative landings for the trimester.  The starting location was used to assign a geographic position to each tow.  Comparisons among the ELB, paper logbooks, and observer records have shown that this methodology provides an accurate representation of the spatial distribution of shrimping effort \cite{gallaway_development_2001,gallaway_development_2000,gallaway_description_2003}.  The initial years of the ELB program instrumented a small number of vessels and were considered pilot studies \cite{gallaway_evaluation_2003,gallaway_description_2003}.    We used ELB data collected over 6 years (2005-2010) and consisting of 411,302 tow records including a date, tow duration (hours), and starting latitude and longitude (Table 1).  While the ELB program currently covers a small fraction of the shrimp fleet in the northern Gulf, the random selection of vessels and extensive spatial coverage (Supp. Fig 1 ) suggest it is representative of fleet-wide patterns in shrimping effort.  

\subsection*{Environmental Data}
Bottom environmental conditions in the northwestern Gulf were characterized using data from the Southeast Area Monitoring and Assessment Program (SEAMAP) survey \cite{eldridge_southeast_1988}.  The SEAMAP survey is a fishery-independent trawl and hydrographic survey that has been conducted semi-annually in the northwestern Gulf  (-88 to -97.3 longitutde and 3-100 $m$ depth) since 1987.  Bottom environmental conditions (e.g., temperatures, salinity, dissolved oxygen, depth) are measured at point locations chosen based on a stratified random sampling design using a conductivity, temperature, depth profiler (CTD).  Prior studies have used these data to quantify spatial patterns in bottom dissolved oxygen and environmental associations with multiple demersal species \cite{craig_spatial_2005,craig_hypoxia-induced_2005}. Additional details regarding the SEAMAP survey and sampling procedures can be found in Eldridge \cite{eldridge_southeast_1988} and the annual SEAMAP atlases \cite{rester_seamap_2011}.  

We generated interpolated surfaces of water depth ($m$) and summer (June-July) bottom dissolved oxygen ($mg L^{-1}$) over the northwestern Gulf of Mexico shelf. Bottom temperature and salinity were considered in preliminary analyses but were removed due to high correlations with water depth.  Following earlier studies \cite{craig_spatial_2005}, bottom water DO was interpolated for each year of ELB data (2005-2010) using universal kriging with a quadratic drift component, and a variable search radius with a sample count of at least 12 nearest neighbors.  Water depth was interpolated using inverse distance weighting of all SEAMAP records that temporally correspond to the ELB timeseries ($n = 7200$). 

Due to the high spatial variability of DO on the coastal shelf and because the directionality and operational area of a tow are unknown we conducted a spatial smoothing procedure  of the interpolated DO surfaces to help account for variation in DO across the area likely covered by individual tows.  Interpolated surfaces were post-processed using the Focal Statistics geoprocessing tool (ESRI, Redlands, CA), to calculate an average interpolation value for each grid cell based on a defined spatial neighborhood.  We defined a square 3x3 neighborhood matrix of 5 $km^{2}$ cells.  For each cell in the interpolated surface the predicted DO values were averaged across the surrounding spatial neighborhood covering a 44 $km^{2}$ area.  Contours for final predicted dissolved oxygen layers were visually assessed for agreement with point measurements of  bottom DO.  All interpolations were conducted in ArcMap v10.1 using the Spatial Analyst or Neighborhood toolbox (ERSI, Redlands, CA).  
\subsection*{Data set Integration}
The objective of integrating the ELB and SEAMAP data sets was to create a composite database in which individual shrimp tows could be matched spatially and temporally with bottom DO and depth values. A challenge in integrating disparate datasets is the different spatial and temporal scales at which data are collected.  Environmental data was assigned to each fishing effort record by sampling the interpolated environmental surface layers based on the spatial start locations of each towing effort.  The summer SEAMAP cruise, from which data for the dissolved oxygen interpolations derive, occurs over a period of 6-8 weeks in regions across the northwestern Gulf.  The challenge in this approach is to match the fishing effort to the environmental sampling with as much spatial and temporal accuracy as possible without sacrificing excessive sample size.  Effort was matched spatially to a resolution of $1\,^{\circ}$ longitude along the northern Gulf coast and $1\,^{\circ}$ latitude along the western Gulf coast, this resolution matches the historical designation for data collection within the shrimp fishery \cite{nance_effort_1993} and the weekly spatial operations of the SEAMAP survey \cite{rester_seamap_2011}.

We examined a number of different temporal resolutions for dataset integrations to find the best compromise between data quantity and temporal accuracy.  Temporal integration of the datasets was complicated by the Texas fishing closure \cite{cody_texas_1989,klima_review_1982,matlock_did_2010,nance_feasibility_1994}, which prevents fishing efforts in Texas near shore and offshore waters from mid-May to mid-July, a period during which much of the summer SEAMAP sampling in Texas waters takes place.  The most temporally concurrent data integration was at a resolution of 1 week; however at this resolution only a fraction of Texas coastal waters fishing effort could be spatially matched.  The temporal mismatch between the Texas fishery closure and the spatial sampling of the SEAMAP survey functionally eliminates data points in southern Texas waters at temporal resolutions less than 6 weeks.  Therefore, we focus on a 6 week temporal integration between the fishery-dependent and fishery-independent data sets for our analysis.

  Due to the significant role of economic drivers in dictating fishing behavior \cite{van_putten_theories_2012} we included two economic based covariates: shrimp market price per pound ($pPND$) and average weekly diesel fuel price ($pGAL$).  The market $pPND$ of shrimp is maintained by NMFS Galveston laboratory as part of the shrimp fishery monitoring program since its inception.  We also used a diesel fuel time series (Ultra-low sulfur CARB diesel spot price) that is maintained by the U. S. Energy Information Administration (http://www.eia.gov).  This fuel time series consists of daily spot prices in USD for diesel fuel in Los Angeles, CA and runs from mid-April 1996 through today.  While spatially there is a more regional fuel dataset that is focused on the Gulf of Mexico region, it is temporally limited to later than June 2006.  We choose the Los Angeles time series because it covered the entire extent of our modeling period and had a 98\% correlation with the Gulf regional diesel price data. 

\subsection*{Regression Models}
We evaluated  the effect of water depth and bottom DO  on the spatial distribution  of shrimping effort using regression analysis of the composite ELB and SEAMAP data sets.  We used a quasi-experimental framework in which separate models were generated for the Louisiana shelf (> -94� longitude), a region that experiences recurrent, severe bottom water hypoxia during summer, and the Texas shelf (< -94� longitude) shelf, a region that experiences limited hypoxia \cite{rabalais_beyond_2002,turner_gulf_2008}.  We considered  three response variables: total shrimping effort, average tow duration, and tow density, each calculated  within a 10x10 minute spatial grid  overlaid across the northwestern Gulf of Mexico.  Environmental covariates were averaged within each spatial grid for all effort instances with a cell.  While total shrimping effort, or total tow duration  within the 10x10 minute grid cells, was our primary metric of fishery effort we also applied our model framework to two additional response variables: average tow length , and tow density, because the specific spatial location of fisheries landings are proprietary and therefor it was impossible for us to attach tow specific landings to our effort data. These secondary response variables allow us to indirectly evaluate fishery behavior with respect to environmental heterogeneity.   

We used generalized additive models (GAMs) to investigate relationships between shrimping effort and environmental variables (DO, depth) using both a spatial and non-spatial model formulation \cite{hastie_varying-coefficient_1993,wood_stable_2004,wood_generalized_2006} structure in which we modeled aggregated effort measurements for both the Louisiana and Texas coastal shelves .  All models were structured with a dependent variable representing a measure of shrimping effort, $X_{d,y,(\rho,\varphi)}$ on Julian day ($d$) in year ($y$) for a specific grid cell with a centroid location ($\rho,\varphi$).  Two of the response variables, total shrimping  effort and average tow duration, were log+1 transformed and modeled with a Gaussian error distribution, while average tow density was natively modeled assuming  a Poisson error distribution.  

\begin{equation}
\begin{aligned}
X_{d,y,(\rho,\varphi)}& = \alpha_1\,(y)\, +\, \alpha_2\,(pGAL)\, +\, \alpha_3\,(totEFF)\,+\, g_1\,(DO)\,  \\
                      &\,+\, g_2\,(D)\, +\, g_3\,(pPND)\, +\, g_4\,(JD)\,+\, g_5\,(\rho,\varphi)\, +\, e_{d,y,(\rho,\varphi)}
\end{aligned}
\end{equation}

The non-spatial model (Eq. 1), dissolved oxygen ($DO$) in $mg{} L^{-1}$, depth ($D$) in meters and market price per pound ($pPND$) and average weekly price per gallon ($pGAL$) in US dollars ($USD$), are 1- dimensional smoothing functions (Wood 2004) while the grid cell centroid location ($\rho,\varphi$) is a 2-dimensional smoothing function (Wood 2003).  The parametric variables $\alpha_1$,$\alpha_2$  and $\alpha_3$ are model  coefficients for the year ($yr$), total annual shrimping effort ($totEFF$), and average weekly fuel price ($pGAL$), and $e_{d,y,(\rho,\varphi)}$ is a model specific error term .  This model form assumes no variability in the spatial effects of the  $DO$ on the distribution of fishing effort.  

\begin{equation}
\begin{aligned}
X_{d,y,(\rho,\varphi)}& = \alpha_1\,(y) + \alpha_2\,(pGAL) + \alpha_3\,(totEFF) + g_1\,(D) + g_2\,(pPND) \\
                      &\,+\, g_3\,(JD)\, +\, g_4\,(\rho,\varphi)\, +\, g_5\,(\rho,\varphi)\,DO\, +\, e_{d,y,(\rho,\varphi)}
\end{aligned}
\end{equation}

The spatial model formulation (Eq. 2) differs  from the stationary formulation by adding an additional term, which is a 2-dimensional smoothing function in which the coefficient, dissolved oxygen ($DO$), is permitted to vary smoothly as a function of space ($\rho,\varphi$).  Variable coefficient models are a class of regression model that allow coefficients to vary smoothly as a function of another model variable \cite{hastie_varying-coefficient_1993}.  Variable-coefficient models that map variation in the relationship between a response and predictor variable across space are often referred to as geospatial regression models \cite{wood_generalized_2006} and have recently be applied to a number of ecological studies \cite{bacheler_spatial_2010,bartolino_ontogenetic_2010,ciannelli_phenological_2007,ciannelli_non-additive_2012}.  All  regression models were conducted in the R 2.14.2 statistical computing environment using the R package mgcv 1.7-13.  


% Do NOT remove this, even if you are not including acknowledgments
\section*{Acknowledgments}


%\section*{References}
% The bibtex filename
\bibliography{SpatialEffortStudy}

\section*{Figure Legends}
%\begin{figure}[!ht]
%\begin{center}
%%\includegraphics[width=4in]{figure_name.2.eps}
%\end{center}
%\caption{
%{\bf Bold the first sentence.}  Rest of figure 2  caption.  Caption 
%should be left justified, as specified by the options to the caption 
%package.
%}
%\label{Figure_label}
%\end{figure}


\section*{Tables}
%\begin{table}[!ht]
%\caption{
%\bf{Table title}}
%\begin{tabular}{|c|c|c|}
%table information
%\end{tabular}
%\begin{flushleft}Table caption
%\end{flushleft}
%\label{tab:label}
% \end{table}

\end{document}

